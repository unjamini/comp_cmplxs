\section{Задача 1}
\subsection{Задание}

Рассмотреть пример из лекционного материала. Построить рабочий список и график сужения интервала. 

\subsection{Решение}
Возьмём функцию Растригина.
\[f(x, y) = x^2 + y^2 - \cos(18*x) - \cos(18*y)\]

Глобальный минимум этой функции достигается в точке \((0, 0)\), значение -2. 

Интервал, полученный при применении метода поиска оптимального интервала:
\[[1e^{-6} * 0.1, 1e^{-6} * 0.1]\]
\[[1e^{-6} * 0.1, 1e^{-6} * 0.1]\]

Построим рабочий список:
\begin{figure}[h]
\caption{Рабочий список для функции Растригина}
\includegraphics[width=12cm]{fig/worklist.png}
\centering
\end{figure}

Построим 2 графика сужения интервала. Красный график - это график сужения интервала по вертикали, а голубой - по горзонтали.

\begin{figure}[h]
\caption{Графики сужения интервала}
\includegraphics[width=12cm]{fig/intervals.png}
\centering
\end{figure}

\section{Задача 2}
\subsection{Задание}
Применить метод для другой функции и исследовать сходимость метода.
Применим метод для Booth function.
\[f(x,y) = \left( x + 2y -7\right)^{2} + \left(2x +y - 5\right)^{2}\]

\subsection{Решение}

На выходе алгоритма: минимальное значение 0 достигается на интервале:
\[[1, 1]\]
\[[3, 3]\]

Построим рабочий список:
\begin{figure}[H]
\caption{Рабочий список для функции Бута}
\includegraphics[width=12cm]{fig/worklistB.png}
\centering
\end{figure}

Графики сужения интервала:
\begin{figure}[H]
\caption{Графики сужения интервала}
\includegraphics[width=12cm]{fig/intervalsB.png}
\centering
\end{figure}

\section{Литература и ссылки}

Репозиторий с кодом программы и кодом отчёта:
\href{https://github.com/unjamini/comp_cmplxs}{https://github.com/unjamini/comp\_cmplxs}

Функции, используемые для оптимизации:
\href{https://en.wikipedia.org/wiki/Test_functions_for_optimization}{TestFunctions}

Презентации лекций:
\href{https://cloud.mail.ru/public/5CaW/TsKmuyMp3/topic3.pdf}{лекции}

Код функции для оптимизации:
\href{http://www.nsc.ru/interval/Programing/MCodes/globopt0.m}{GlobOpt0}
